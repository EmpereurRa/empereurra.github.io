\documentclass[12px]{article}

\usepackage{titlesec}
\usepackage{titling}
\usepackage[margin=1.25in]{geometry}
\usepackage[utf8]{inputenc}
\usepackage[backend=bibtex, style=numeric, sorting=none]{biblatex}
\addbibresource{references.bib}

\titleformat{\section}{\huge\bfseries}{\thesection}{.25em}{}[\titlerule]
\titleformat{\subsection}{\bfseries\Large}{\hspace{-.25in}$\bullet$}{0em}{}
\titleformat{\subsubsection}[runin]{\bfseries}{}{0em}{}[---]
\titlespacing{\subsubsection}
{1em}{.25em}{1em}

\title{States of Matter and interesting variables}
\author{Ra}

\newcommand{\aparagraph}[1]{\paragraph{#1}\mbox{}\\}

\renewcommand{\maketitle}{
        \begin{center}

        {\huge\bfseries \thetitle}
        \newline

        \vspace{.25em}

        \end{center}
}

\begin{document}
  \maketitle
  \begin{center}
	  \begin{tabular}{|c|c|c|c|c|}
	  \hline
		  \textbf{Object} & \textbf{State of Matter} & \textbf{Average Time to Live} & \textbf{Climate}  & \textbf{Object Cost} \\
	  \hline
		  Water vapor & Gas &  not appliable & (usually) hot & not applicable \\
	  \hline
		  orange juice & liquid & 1-3 days \parencite{orangejuicelife} & any & 2.89\$/KG \parencite{orangejuicecost}  \\
	  \hline
		  a house's walls & solid & ~100 years \parencite{glasslife} & any & Bricks: 56\$–330\$/m2 \parencite{wallcost} \\
	  \hline
		  Fire & Plasma* & a few hours & any & cost of materials used \\
	  \hline
		  Glass & Solid* & 15 years \parencite{glasslife} & any & 10\$/square foot \parencite{glasscost} \\
	  \hline

          \end{tabular}
  \end{center}

\aparagraph{I Object: \\
- The Object described by the other fields}
\aparagraph{II State of Matter notes: \\
We will use the following states of matter for this research: \\
   a) Solid: Closely packed particles  with strong inter-molecular forces leading to particles able to only vibrate but not move freely. A solid has a stable, definite shape and a definite volume \parencite{solids}; \\
   b) Liquid: A liquid is a nearly incompressible fluid that conforms to the shape of its container but retains a (nearly) constant volume independent of pressure \parencite{liquids}; \\
   c) Gas: a compressible fluid. Not only will a gas conform to the shape of its container but it will also expand to fill the container \parencite{gases} ; \\
   d) Plasma: Ionized gaz with electrons \parencite{plasma}. }
\aparagraph{III Average Time to Live: \\
   - Average living time for an object, living time = time existing in a form that is not decomposed or altered beyond resemblence of the original object.}
\aparagraph{IV Climate notes: \\
1. for usage of x object at specific temperatures we will use the following notation: \\
   a) very cold = average temperature of place of usage under -10 degrees Celsius; \\
   b) cold = average temperature of place of usage from -10 degrees to 10 degrees Celsius; \\
   c) medium = average temperature of place of usage is from 10 degrees to 25 degrees Celsius; \\
   d) hot = average temperature of place of usage is from 25 degrees to 40 degrees. \\
2. for something used at least in temperatures from -20 degrees Celsius to 40 degrees Celsius we will use the notation "any".}
\aparagraph{V Object cost:  \\
   - Average cost of an object worldwide.}
\aparagraph{*notes: \\
 1. Fire is either considered a low-level plasma or a lightly ionized gas. \parencite{firedef}; \\
 2. Glass is a non-crystaline (it lacks long-range symmetry in it's system of particles, that which is characteristic of a crystal) solid that is sometimes considered to be a liquid due to its lack of a first-order phase transition where certain thermodynamic variables are discontinuous through the glass transition range. \parencite {glassdef} }

\printbibliography

\end{document}
