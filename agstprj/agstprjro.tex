\documentclass[12px]{article}

\usepackage{titlesec}
\usepackage{titling}
\usepackage[margin=1.25in]{geometry}
\usepackage[utf8]{inputenc}
\usepackage[backend=bibtex, style=numeric, sorting=none]{biblatex}
\addbibresource{references.bib}

\titleformat{\section}{\huge\bfseries}{\thesection}{.25em}{}[\titlerule]
\titleformat{\subsection}{\bfseries\Large}{\hspace{-.25in}$\bullet$}{0em}{}
\titleformat{\subsubsection}[runin]{\bfseries}{}{0em}{}[---]
\titlespacing{\subsubsection}
{1em}{.25em}{1em}

\title{Stari de Agregare si Variabile interesante}
\author{Ra}

\newcommand{\aparagraph}[1]{\paragraph{#1}\mbox{}\\}

\renewcommand{\maketitle}{
        \begin{center}

        {\huge\bfseries \thetitle}
        \newline

        \vspace{.25em}

        \end{center}
}

\begin{document}
  \maketitle
  \begin{center}
	  \begin{tabular}{|c|c|c|c|c|}
	  \hline
		  \textbf{Obiectul} & \textbf{Stare de Agregare} & \textbf{Timp de viata mediu} & \textbf{Clima}  & \textbf{Costul Obiectului} \\
	  \hline
		  Vapor de apa & Gaz &  ne-aplicabil & (deobicei) cald & ne-aplicabil \\
	  \hline
		  suc de portocale & lichid & 1-3 zile \parencite{orangejuicelife} & oricare & 2.89\$/KG \parencite{orangejuicecost}  \\
	  \hline
		  Peretii unei case & solid & ~100 ani \parencite{glasslife} & oricare & Caramizi: 56\$–330\$/m2 \parencite{wallcost} \\
	  \hline
		  Focul & Plasma* & Cateva ore & oricare & costul materialelor folisite.\\
	  \hline
		  Sticla & Solid* & 15 ani \parencite{glasslife} & oricare & 10\$/picior patrat \parencite{glasscost} \\
	  \hline

          \end{tabular}
  \end{center}

\aparagraph{I Obiectului: \\
- Obiectul descris de celelalte variabile}
\aparagraph{II Stare de Agregare: \\
Vom folosi urmatorile stari de agregare pentru cercetare \\
   a) Solid: Particule stranse impreuna puternic cu forte inter-moleculare ducand la particule capabile doar de a vibra dar nu de a se misca liber. Un solid are o forma stabila si definita, si un volum definit \parencite{solids}; \\
   b) Lichid: Un lichid este aproape incompresibil comformanduse formei container-ului dar retinant un volum (aproape) constant independent de presiunel \parencite{liquids}; \\
   c) Gaz: un fluid compresibil, nu doar se va obisnui cu forma contain-erul dar se si extinde sa controleze cat mai mult spatiu. \parencite{gases} ; \\
   d) Plasma: Gas ionic cu electroni \parencite{plasma}. }
\aparagraph{III Timp de viata mediu: \\
   - Timp de viata mediu, timp de viata = timp existand intr-o forma care nu este decompusa sau prea departe de forma originala.}
\aparagraph{IV Clima: \\
1. Pentru utilizarea obiectului x la o temperatura specific vom folosi urmatoarea notatie: \\
   a) foarte rece = temperatura medie de -10 grade Celsius pentru zona de utilizare; \\
   b) rece = temperatura medie de folosire de la -10 Grade la 10 Grade Celsius; '' \\
   c) mediu = temperatura medie de folosire de la -10 grade la 25 de grade; \\
   d) cald = temperatura medie de folosire de la -25 Grade Celsius la 40 grade celsius. \\
2. Pentru ceva folosit cel putin in temperaturi de la -20C la 40 Celsius vom folosi notatia "oricare".}
\aparagraph{V Costul Obiectului:  \\
   - Cost mediu al unui obiect.}
\aparagraph{*notite: \\
 1. Focul este cateodata considerat plasma de nivel jos, cateodata un gaz putin ionizat. \parencite{firedef}; \\
 2. Sticla este un solid non-cristalin (ii lipseste simetrie "long-range" in sistemul sau de particule, care este o caracteristica a cristalelor) care este cateodata considerat sa fie un lichid datorta lipsi unei prim-ordin tranzitie de faza unde anumite variabile termodinamice sunt discontinue prin raza de tranzitie a sticlei. \parencite {glassdef} }

\printbibliography

\end{document}
